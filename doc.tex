\documentclass[11pt]{article}
\usepackage[T1]{fontenc}
\usepackage[swedish]{babel}

\usepackage[utf8x]{inputenc}

\usepackage{fancyvrb}
\usepackage{amssymb}

% Minskar margins
\usepackage{fullpage}


\title{ETIA01}
\author{Meris Bahtijaragic\\Bengt Ericsson}
\date{2013--02--07}

\begin{document}
\maketitle

\section{Förstärkare}
\subsection{Begrepp}
\begin{itemize}
  \item \textbf{Differential Mode} Skillnaden mellan signalerna. ($V_{dm}$)
  \item \textbf{Common Mode} Medelvärdet av signalerna. ($V_{cm}$)
  \item \textbf{Common Mode Gain} Förstärkningen av medelvärdet av signalerna. ($V_{cm}$)
  \item \textbf{Differential Mode Gain} Förstärkningen av skillnaden mellan signalerna. ($V_{dm}$)
  \item \textbf{CMMR} Common Mode Rejection Ratio, förstärkarens förmåga att undertrycka den gemensamma signalen. Anges ibland i dB. ($\frac{A_{dm}}{A_{cm}}$)
\end{itemize}

\section{AD-omvandlare}
\subsection{Karaktärsdrag}
\begin{tabular}{ |l|l|l| }
  \hline
  AD-Omvandlare & Snabbhet & Upplösning \\
  \hline
  Flash & ++++ & + \\
  Successiv Approximation & +++ & ++ \\
  Sigma Delta & ++ & +++\\
  Integrerande & + & ++++ \\
  \hline
\end{tabular}

\subsection{Typisk Kretsmodell}
%Sätt in figur här, ritad av Bengt himself.

\subsection{Termer}
\textbf{Upplösning}: Den minsta ändring i spänning som AD-omvandlaren kan mäta av.

\subsection{Generell Metod För Analys av Förstärkare}
Kolla först så att det är negativ återkoppling.
Använd att $v_{in+} - v_{in-} = 0$, $i_{in} = 0$ och $R_{ut} = 0$.
Märk hur strömmen går.
Använd KVL och KCL för att lösa ut $V_{ut}$.

Exempel:
%Figur ritad av Bengt himself:
Utför KVL för slinga in i $V_{in}$, sen genom $R_{a}$, sen till jord genom förstärkaren
(notera att det är 0V spänningsfall i mellan förstärkarens inportar).
Resultat: $V_{in} = V_a$

Utför KVL på slinga från förstärkarens pluspol, genom förstärkaren, till RB, slutligen ut i $V_{ut}$.
Resultat: $V_{b} = -V_{ut}$.

$V_{b} = I_{a} * R_b = \frac{V_{in}}{R_a} * R_b = \frac{R_b}{R_a} * V_{in}$

\subsection{Filter}
\subsection{Lågpass}
%Figur ritad av bengt himself.
Tänk dig en figur med en seriekopplad resistans och en kapacitator, man tar ut spänningen över kondensatorn.
Plant $0 dB$ t.om brytfrekvensen sen sen $-20dB/dekad$.
På formen $H(j\omega) = \frac{1}{1+j\omega RC}$

\subsection{Högpass}
%Figur ritad av bengt himself.
Tänk dig en figur med en seriekopplad resistans och en kapacitator, man tar ut spänningen över resistansen.
På formen $H(j\omega) = \frac{1}{1-j\frac{1}{\omega RC}}$
$+20dB/dekad$ t.om brytfrekvensen sen planar det ut på $0 dB$

\end{document}
